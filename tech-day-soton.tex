\documentclass[12pt,a4paper]{scrartcl}
\usepackage{url, booktabs,pdflscape}
\title{Student Robotics Tech Day Risk Assessment Form}

\begin{document}
\maketitle

\begin{description}
\item[Activity being assessed:] Student Robotics Tech Day 2012 (03/12/2011, 04/02/2012, 17/03/2012)
\item[Location:] Room 1009, Building 25, Highfield Campus, University of Southampton (\url{http://data.southampton.ac.uk/building/25.html})
\item[Who is exposed to the hazard:] Competitors, Teachers, Mentors
\end{description}

\begin{description}
\item[Assessor's name:]
\item[Assessor's job title:]
\item[Assessor's signature:]
\item[Date of assessment:]
\item [Reviewed as acceptable by accountable committee member]
\item[Reviewer's name:]
\item[Reviewer's job title:]
\item[Reviewer's signature:]
\end{description}
\clearpage

\newcommand{\risk}[3]{
 #1 & #2 & #3 \\
}

\begin{landscape}
\section{Risks}
The following risks have been considered for the student robotics competition.  Further description of the meaning of risk ratings (presented in this section as $L \times S$) can be found in the next section.

A safety briefing will be given on all mornings, covering the points below.

\bigskip
\begin{tabular*}{\linewidth}[c]{p{14em}p{30em}c}
\toprule
\textbf{Hazard} & \textbf{Control Measures} & \textbf{Risk Rating} \\
\midrule

\risk{Injury while using power or manual tools}
{Teachers to supervise all use of tools given that teams bring their own. These tools are used at the teams own risk. Student Robotics will not provide tools at any Tech Day.}
{3}

\risk{Electrocution by contact between water, electrical output and human}
{Water and electrical outputs kept strictly apart. Food and Drink is not allowed in the pit areas (i.e. places where teams work on their robots), or around the mini-arena.}
{3}

\risk{Risk of Fire}
{No naked flames are allowed to be used intentionally. If a fire breaks out accidentally, iSolutions regulations will be followed as described here: \url{http://www.southampton.ac.uk/isolutions/essentials/learnandteach/cls/fire.html}}
{2}

\risk{Interaction with robots: electric shock, minor injury.}
{Competitors are only allowed into the mini-arena with a mentor present, and may only work on robots in their pits. Robots may only be tested under supervision and if robot safety rules are met (see rulebook 2.10-2.14, 2.16, 2.17). Electronics provided by Student Robotics are housed in a plastic casing, and wiring will be inspected by a member of Student Robotics before competitors are allowed to work on their robots.  Rulebook: \url{https://www.studentrobotics.org/resources/2012/rulebook.pdf}}
{1}

\risk{Misuse of batteries}
{See rules 2.16, 2.17. Batteries must only be charged as described here:
\url{https://www.studentrobotics.org/docs/kit/batteries}}
{2}
\bottomrule
\end{tabular*}
\end{landscape}

\begin{landscape}

\section{Assessment Guidance}

The risk ratings of the risks in the previous section are calculated by multiplying $L$, the likelihood rating, by $S$, the severity rating.

\bigskip
\begin{minipage}[b]{0.5\linewidth}
\begin{tabular}[c]{lc}
  \toprule
  \textbf{Likelihood} & \textbf{Likelihood rating} \\
  \midrule
  Very unlikely & 1 \\
  Unlikely & 2 \\
  Likely & 3 \\
  Fairly likely & 4 \\
  Very likely & 5 \\
  \bottomrule
\end{tabular}
\end{minipage}
\begin{minipage}[b]{0.5\linewidth}
\begin{tabular}[c]{lc}
  \toprule
  \textbf{Severity} & \textbf{Severity rating} \\
  \midrule
  First Aid injury/illness & 1 \\
  Minor injury/illness & 2 \\
  `3 day' injury/illness & 3 \\
  Major injury/illness & 4 \\
  Fatality/disabling injury & 5 \\
  \bottomrule
\end{tabular}
\end{minipage}
\bigskip

The following should be used to rate the risk and plan corrective action:
\bigskip
\newcommand{\riskinfo}[4]{
  #1 & #2 & #3 & #4 \\
}

\begin{tabular*}{\linewidth}[c]{cccp{33em}}
  \toprule
  \textbf{Risk Rating} & \textbf{Category} & \textbf{Tolerability} & \textbf{Comments} \\
  \midrule

  \riskinfo{1--2}{Very Low}{Acceptable}
  {No further action is necessary other than to ensure that the controls are maintained.}

  \riskinfo{3--4}{Low}{Acceptable}
  {No additional controls are required unless they can be implemented at very low cost (in terms of time, money and effort).}

  \riskinfo{5--7}{Medium}{Tolerable}
  {Consideration should be given as to whether the risks can be lowered, where applicable, to a tolerable level, and preferably acceptable level, but the costs of additional risk reduction measures should be taken into account.  The risk reduction measures should be implemented within a defined time period.}

  \riskinfo{8--14}{High}{Tolerable}
  {Substantial efforts should be made to reduce the risk.  Risk reduction measures should be implemented urgently within a defined time period and it might be necessary to consider suspending or restricting the activity, or to apply interim risk control measures, until this has been completed. Considerable resources might have to be allocated to additional control measures.}

  \riskinfo{15 and above}{Very High}{Unacceptable}
  {Substantial improvements in risk control are necessary, so that risk is reduced to a tolerable or acceptable level.}

  \bottomrule
\end{tabular*}

\end{landscape}

\clearpage
\appendix
\section{Fire Safety}
\textit{From iSolutions Regulations -- \url{http://www.southampton.ac.uk/isolutions/essentials/learnandteach/cls/fire.html}}

All Common Learning Spaces have a Fire Evacuation Route Poster located usually near the exit of in a glassed wall display cabinet alongside the other Common Learning Spaces signage.

\subsection{If you discover a fire}
\begin{enumerate}
\item Activate the alarm at any fire alarm call point by breaking the glass.
\item Evacuate the building by the most direct route.
\item Report to the assembly area
\end{enumerate}

\subsection{If you hear the alarm}
\begin{enumerate}
\item  Switch off any electrical equipment that you have been using, if safe to do so.
\item Close the door of the room when leaving.
\item Evacuate the building by the most direct route, and report to the assembly point.
\end{enumerate}

\end{document}

